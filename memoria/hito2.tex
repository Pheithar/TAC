\documentclass{uc3mpracticas}

\usepackage{helvet}
\usepackage{multicol}
\renewcommand{\familydefault}{\sfdefault}
\usepackage{changepage}
\usepackage{geometry}
\usepackage{caption}
\usepackage{xcolor,colortbl}
\usepackage{makecell}

\usepackage{amsfonts}

\definecolor{Gray}{gray}{0.85}
\definecolor{LightCyan}{rgb}{0.88,1,1}
\definecolor{LightGreen}{rgb}{0.29,1,0.39}

\newcolumntype{g}{>{\columncolor{Gray}}l}
\newcolumntype{b}{>{\columncolor{LightCyan}}c}


%%%%%%%%%%%%%%%%%%%%%%%%%%%%%%%%%%%%%%%%%%%%%%%%%%%%%%%%%%%%%%%%%%%%%%%%%%%%%%%%
%%%                   Plantilla Prácticas UC3M                               %%%
%%%                Universidad Carlos III de Madrid                          %%%
%%%                   Alejandro Valverde Mahou                               %%%
%%%%%%%%%%%%%%%%%%%%%%%%%%%%%%%%%%%%%%%%%%%%%%%%%%%%%%%%%%%%%%%%%%%%%%%%%%%%%%%%

%Permitir cabeceras y pie de páginas personalizados
\pagestyle{fancy}

%Path por defecto de las imágenes
\graphicspath{ {./images/} }

%Declarar formato de encabezado y pie de página de las páginas del documento
\fancypagestyle{doc}{
  %Cabecera
  \headerpr[1]{Test de Primalidad - AKS}{Hito 2}{Teoría Avanzada de la Computación}
  %Pie de Página
  \footerpr{}{\textbf{UC3M}}{{\thepage} de \pageref{LastPage}}
}

%Declarar formato de encabezado y pie del título e indice
\fancypagestyle{titu}{%
  %Cabecera
  \headerpr{}{}{}
  %Pie de Página
  \footerpr{}{}{}
}


\appto\frontmatter{\pagestyle{titu}}
\appto\mainmatter{\pagestyle{doc}}


\begin{document}
  %Comienzo formato título
  \frontmatter


  %Portada 1 (Centrado todo)
  \centeredtitle{Images/LogoUC3M.png}{Grado en Ingeniería Informática}{Curso 2020/2021}{Teoría Avanzada de la Computación}{Test de Primalidad - \textit{AKS}}{Hito 2}

  \vspace{55mm}

  \authors{Iván Miguelez García}{100383387}{Alba Reinders Sánchez}{100383444}{Alejandro Valverde Mahou}{100383383}{}{}

  \newpage

  %Índice
  \tableofcontents

  \newpage

  %Comienzo formato documento general
  \mainmatter

  \section{Hito 2: Estudio de la complejidad del \textit{Totient}}

  En esta segunda parte se pide realizar el estudio analítico y empírico del cálculo del \textit{Totient} ($\phi(r)$). Donde $\phi(r)$ es el número de enteros positivos más pequeños o iguales que $r$ tales que $r$ es coprimo con ellos, es decir, su $mcd$ es 1.



  \subsection{Estudio analítico}

  A continuación, se realiza el estudio analítico para averiguar la complejidad temporal del algoritmo. Analizando el código se ve que la función del cálculo del \textit{Totient} está formada por un bucle \textit{for} externo y un bucle \textit{while} interno:

  \vspace{2mm}

  El bucle de fuera se ejecuta como mucho $\lfloor \sqrt{n} \rfloor$ veces, dado que en este caso, la peor situación se da cuando $n$ es un número primo, y por tanto el bucle de fuera tiene que recorrer desde $i = 2$ hasta $i = \lfloor \sqrt{n} \rfloor + 1$. Simplificando, se encuentra $\sqrt{n}$ veces.

  \vspace{2mm}

  El bucle de dentro se ejecuta como mucho $\log_i{n}$ veces, porque el peor caso resulta cuando $ n = i^k $ donde $ k $ es un número entero. Por tanto, despejando, $k = \log_in$, y $k$ representa el número de veces que se realiza el bucle.

  \vspace{2mm}

  Para que se cumpla el peor de los casos del bucle \textit{for} exterior, $n$ tiene que ser un número primo, y por tanto el bucle \textit{while} interior no se realizará ninguna vez. En el caso de que $n$ sea el número primo, también se obtiene el valor máximo de $\phi(n)$, que es $n-1$.



  \subsection{Estudio empírico}



% \newpage
%
% \bibliographystyle{unsrt}
% \bibliography{bibliography}

\end{document}
