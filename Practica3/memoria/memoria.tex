\documentclass{uc3mpracticas}

\usepackage{helvet}
\usepackage{multicol}
\renewcommand{\familydefault}{\sfdefault}
\usepackage{changepage}
\usepackage{geometry}
\usepackage{caption}
\usepackage{xcolor,colortbl}
\usepackage{makecell}
\usepackage{mathtools}

\usepackage{amsfonts}

\definecolor{Gray}{gray}{0.85}
\definecolor{LightCyan}{rgb}{0.88,1,1}
\definecolor{LightGreen}{rgb}{0.29,1,0.39}

\newcolumntype{g}{>{\columncolor{Gray}}l}
\newcolumntype{b}{>{\columncolor{LightCyan}}c}


%%%%%%%%%%%%%%%%%%%%%%%%%%%%%%%%%%%%%%%%%%%%%%%%%%%%%%%%%%%%%%%%%%%%%%%%%%%%%%%%
%%%                   Plantilla Prácticas UC3M                               %%%
%%%                Universidad Carlos III de Madrid                          %%%
%%%                   Alejandro Valverde Mahou                               %%%
%%%%%%%%%%%%%%%%%%%%%%%%%%%%%%%%%%%%%%%%%%%%%%%%%%%%%%%%%%%%%%%%%%%%%%%%%%%%%%%%

%Permitir cabeceras y pie de páginas personalizados
\pagestyle{fancy}

%Path por defecto de las imágenes
\graphicspath{ {./images/} }

%Declarar formato de encabezado y pie de página de las páginas del documento
\fancypagestyle{doc}{
  %Cabecera
  \headerpr[1]{Máquinas de Turing}{}{Teoría Avanzada de la Computación}
  %Pie de Página
  \footerpr{}{\textbf{UC3M}}{{\thepage} de \pageref{LastPage}}
}

%Declarar formato de encabezado y pie del título e indice
\fancypagestyle{titu}{%
  %Cabecera
  \headerpr{}{}{}
  %Pie de Página
  \footerpr{}{}{}
}


\appto\frontmatter{\pagestyle{titu}}
\appto\mainmatter{\pagestyle{doc}}


\begin{document}
  %Comienzo formato título
  \frontmatter


  %Portada 1 (Centrado todo)
  \centeredtitle{Images/LogoUC3M.png}{Grado en Ingeniería Informática}{Curso 2020/2021}{Teoría Avanzada de la Computación}{Máquinas de Turing}{}

  \vspace{55mm}

  \authors{Iván Miguélez García}{100383387}{Alba Reinders Sánchez}{100383444}{Alejandro Valverde Mahou}{100383383}{}{}

  \newpage

  %Índice
  \tableofcontents

\newpage

  %Comienzo formato documento general
  \mainmatter

  \section{Búsqueda basada en \textit{Backtracking}}

  Búsqueda basada en Backtracking o equivalente iterativa. Explorar la posibilidad de mejora mediante una heurística sencilla o aplicando criterios de poda. En este caso se busca garantizar siempre una solución óptima. Será difícil encontrar soluciones para problemas de tamaños n mayores a 14 … 20.

  \subsection{Coste Computacional}

  \subsubsection{Estudio Analítico}

  \subsubsection{Estudio Empírico}

  \subsubsection{Estudio Combinado}
  




  \section{Búsqueda local}

  Búsqueda local. Partiendo de la solución de un algoritmo greedy [1], aplicar el operador 2-opt [2]. En este caso, no se podrá garantizar una solución óptima, pero se podrá estudiar qué calidad puede proporcionar el método (en \% de desviación respecto al recorrido óptimo) en un tiempo razonable.

  \subsubsection{Estudio Analítico}

  \subsubsection{Estudio Empírico}

  \subsubsection{Estudio Combinado}



\end{document}
