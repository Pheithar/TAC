\documentclass{uc3mpracticas}

\usepackage{helvet}
\usepackage{multicol}
\renewcommand{\familydefault}{\sfdefault}
\usepackage{changepage}
\usepackage{geometry}
\usepackage{caption}
\usepackage{xcolor,colortbl}
\usepackage{makecell}
\usepackage{mathtools}

\usepackage{amsfonts}

\definecolor{Gray}{gray}{0.85}
\definecolor{LightCyan}{rgb}{0.88,1,1}
\definecolor{LightGreen}{rgb}{0.29,1,0.39}

\newcolumntype{g}{>{\columncolor{Gray}}l}
\newcolumntype{b}{>{\columncolor{LightCyan}}c}


%%%%%%%%%%%%%%%%%%%%%%%%%%%%%%%%%%%%%%%%%%%%%%%%%%%%%%%%%%%%%%%%%%%%%%%%%%%%%%%%
%%%                   Plantilla Prácticas UC3M                               %%%
%%%                Universidad Carlos III de Madrid                          %%%
%%%                   Alejandro Valverde Mahou                               %%%
%%%%%%%%%%%%%%%%%%%%%%%%%%%%%%%%%%%%%%%%%%%%%%%%%%%%%%%%%%%%%%%%%%%%%%%%%%%%%%%%

%Permitir cabeceras y pie de páginas personalizados
\pagestyle{fancy}

%Path por defecto de las imágenes
\graphicspath{ {./images/} }

%Declarar formato de encabezado y pie de página de las páginas del documento
\fancypagestyle{doc}{
  %Cabecera
  \headerpr[1]{Máquinas de Turing}{}{Teoría Avanzada de la Computación}
  %Pie de Página
  \footerpr{}{\textbf{UC3M}}{{\thepage} de \pageref{LastPage}}
}

%Declarar formato de encabezado y pie del título e indice
\fancypagestyle{titu}{%
  %Cabecera
  \headerpr{}{}{}
  %Pie de Página
  \footerpr{}{}{}
}


\appto\frontmatter{\pagestyle{titu}}
\appto\mainmatter{\pagestyle{doc}}


\begin{document}
  %Comienzo formato título
  \frontmatter


  %Portada 1 (Centrado todo)
  \centeredtitle{Images/LogoUC3M.png}{Grado en Ingeniería Informática}{Curso 2020/2021}{Teoría Avanzada de la Computación}{Máquinas de Turing}{}

  \vspace{55mm}

  \authors{Iván Miguélez García}{100383387}{Alba Reinders Sánchez}{100383444}{Alejandro Valverde Mahou}{100383383}{}{}

  \newpage

  %Índice
  \tableofcontents

\newpage

  %Comienzo formato documento general
  \mainmatter


  \section{Introducción}

  En esta memoria se proponen diversas soluciones a los problemas planteados en el enunciado de la práctica, junto con un análisis computacional de cada una de las máquinas de Turing diseñadas.

  \vspace{2mm}

  Estos análisis no siempre se llevan a cabo mediante el cálculo de diferencias finitas, ya que la complejidad de las máquinas no siempre será polinómica. También se realiza una comparación entre las máquinas de una y dos cintas, como deterministas y no deterministas, con el fin de observar si se ha producido alguna mejora en el coste computacional y de qué mejora se trata.


  \newpage

  \section{Palíndromos I}

  \subsection{MT Determinista de 1 cinta}


  \subsubsection{Implementación Propuesta}

  \begin{figure}[!h]
    \imgcenter[150]{Images/ej0a1.png}
    \caption{MT Determinista 1 cinta - Palíndromos}
  \end{figure}




  \subsubsection{Determinación del Peor Caso}

  El peor caso ocurre cuando la entrada es palíndromo con cardinalidad par. Las palabras son de tamaño n=2k, según el espacio definido donde está contenido el conjunto de palíndromos. Cada recorrido completo de la cinta comprueba dos símbolos. Se recorren más símbolos cuando la palabra introducida es un palíndromo, ya que si no lo es, la máquina deja de recorrer la cinta.

  \begin{table}[!h]
    \centering
  \begin{tabular}{|c|c|c|}
  \hline
  \textbf{Entrada} & \textbf{Pasos} & \textbf{Palíndromo} \\ \hline
  aaaa             & 15             & SÍ                  \\ \hline
  aabb             & 9              & NO                  \\ \hline
  aabbaa           & 28             & SÍ                  \\ \hline
  aabaaa           & 27             & NO                  \\ \hline
  babaaa           & 13             & NO                  \\ \hline
  \end{tabular}
  \caption{Peor caso}
  \end{table}



  \subsubsection{Simulación con Diferentes Tamaños}

    \begin{table}[!h]
      \centering
    \begin{tabular}{|c|c|c|}
    \hline
    \textbf{Entrada} & \textbf{Tamaño} & \textbf{Pasos} \\ \hline
    $\lambda$           & 0               & 1              \\ \hline
    aa               & 2               & 6              \\ \hline
    abba             & 4               & 15             \\ \hline
    abaaba           & 6               & 28             \\ \hline
    ababbaba         & 8               & 45             \\ \hline
    ababaababa       & 10              & 66             \\ \hline
    \end{tabular}
    \caption{Diferentes tamaños}
    \end{table}

  \subsubsection{Cálculo de $T(n)$}





\begin{table}[!h]
  \centering
\begin{tabular}{|c|p{1cm}|p{1cm}|p{1cm}|p{1cm}|p{1cm}|p{1cm}|}
\hline
\textbf{N}     & \multicolumn{1}{c|}{\textbf{0}} & \multicolumn{1}{c|}{\textbf{2}} & \multicolumn{1}{c|}{\textbf{4}} & \multicolumn{1}{c|}{\textbf{6}} & \multicolumn{1}{c|}{\textbf{8}} & \multicolumn{1}{c|}{\textbf{10}} \\ \hline
\textbf{Pasos} & \multicolumn{1}{c|}{1}          & \multicolumn{1}{c|}{6}          & \multicolumn{1}{c|}{15}         & \multicolumn{1}{c|}{28}         & \multicolumn{1}{c|}{45}         & \multicolumn{1}{c|}{66}          \\ \hline
\textbf{Diferencia 1}                  &                                                         & 5                                                       & 9                               & 13                              & 21                              & 25                               \\ \hline
\textbf{Diferencia 2}                  &                                                         & \multicolumn{1}{r|}{4}                                  & \multicolumn{1}{r|}{4}          & \multicolumn{1}{r|}{4}          & \multicolumn{1}{r|}{4}          &                                  \\ \hline
\textbf{Diferencia 3}                  &                                                         &                                                         & 0                               & 0                               &                                 &                                  \\ \hline
\end{tabular}
\caption{Diferencias finitas}
\end{table}


  Dado que en la \textit{Diferencia 2} se encuentran valores constantes, es una ecuación de segundo grado:

  $$ T(n) = an^2 + bn + c $$

  Despejando sus valores se obtiene:

  $$ T(0) = c = 1 $$
  $$ T(2) = 4a + 2b + c = 6 $$
  $$ T(4) = 16a + 4b + c = 15 $$

  $$ a = \frac{1}{2} ,\quad b = \frac{3}{2} ,\quad c = 1 $$

  La complejidad de esta máquina de Turing es:

  $$ T(n) = \frac{1}{2}n^2 + \frac{3}{2}n + 1 $$

  Por tanto el valor de la cota asintótica superior $g(n)$ con $n_0 = 10$ es:

  $$ g(n) = kn^2 $$

  $$ g(n) > T(n) $$

  $$ kn^2 > \frac{1}{2}n^2 + \frac{3}{2}n + 1 $$
  $$ k > \frac{1}{2} + \frac{3}{2n} + \frac{1}{n^2}$$
  $$ n_0 = 10 $$
  $$ k > \frac{1}{2} + \frac{3}{2 * 10} + \frac{1}{10^2}$$
  $$ k > \frac{1}{2} + \frac{3}{20} + \frac{1}{100}$$
  $$ k > \frac{66}{100}$$

  $$ k > 0.66 $$
  $$ k = 0.67 $$

  $$ g(n) = 0.67n^2 $$






  \subsection{MT Determinista de 2 cintas}

  \subsubsection{Implementación Propuesta}


  \begin{figure}[!h]
    \imgcenter[150]{Images/ej0b1.png}
    \caption{MT Determinista de 2 cintas - Palíndromos}
  \end{figure}


  \subsubsection{Determinación del Peor Caso}

  En este caso, todos los ejemplos del mismo tamaño tardan lo mismo, independientemente de si son palíndromos o no.


  \begin{table}[!h]
    \centering
  \begin{tabular}{|c|c|c|}
  \hline

  \textbf{Entrada} & \textbf{Pasos} & \textbf{Palíndromo} \\ \hline

  aaaa             & 15             & SÍ                  \\ \hline
  aabb             & 15             & NO                  \\ \hline
  aabbaa           & 21             & SÍ                  \\ \hline
  aabaaa           & 21             & NO                  \\ \hline
  babaaa           & 21             & NO                  \\ \hline
  \end{tabular}
  \caption{Peor caso}
  \end{table}

  \newpage
  \subsubsection{Simulación con Diferentes Tamaños}

  \begin{table}[!h]
    \centering
  \begin{tabular}{|c|c|c|}
  \hline

  \textbf{Entrada} & \textbf{Tamaño} & \textbf{Pasos} \\ \hline

  $\lambda$           & 0               & 4              \\ \hline
  aa               & 2               & 9              \\ \hline
  abba             & 4               & 15             \\ \hline
  abaaba           & 6               & 21             \\ \hline
  ababbaba         & 8               & 27             \\ \hline
  ababaababa       & 10              & 33             \\ \hline
  \end{tabular}
  \caption{Diferentes tamaños}
  \end{table}






  \subsubsection{Cálculo de $T(n)$}

  \begin{table}[!h]
    \centering
  \begin{tabular}{|c|p{1cm}|p{1cm}|p{1cm}|p{1cm}|p{1cm}|p{1cm}|}
  \hline
  \textbf{N}     & \multicolumn{1}{c|}{\textbf{0}} & \multicolumn{1}{c|}{\textbf{2}} & \multicolumn{1}{c|}{\textbf{4}} & \multicolumn{1}{c|}{\textbf{6}} & \multicolumn{1}{c|}{\textbf{8}} & \multicolumn{1}{c|}{\textbf{10}} \\ \hline
  \textbf{Pasos} & \multicolumn{1}{c|}{3}          & \multicolumn{1}{c|}{9}          & \multicolumn{1}{c|}{15}         & \multicolumn{1}{c|}{21}         & \multicolumn{1}{c|}{27}         & \multicolumn{1}{c|}{33}          \\ \hline
  \textbf{Diferencia 1}                  &                                                         & 6                                                       & 6                               & 6                              & 6                              & 6                               \\ \hline
  \textbf{Diferencia 2}                  &                                                         & \multicolumn{1}{r|}{0}                                  & \multicolumn{1}{r|}{0}          & \multicolumn{1}{r|}{0}          & \multicolumn{1}{r|}{0}          &                                  \\ \hline
  \end{tabular}
  \caption{Diferencias finitas}
  \end{table}






  Dado que en la \textit{Diferencia 1} se encuentran valores constantes, es una ecuación de primer grado:


  $$ T(n) = an+ b $$

  Despejando sus valores se obtiene:

  $$ T(0) = b = 3 $$
  $$ T(2) = 2a + b = 9 $$

  $$ a = 3 , \quad b = 3$$

  La complejidad de esta máquina de Turing es:

  $$ T(n) = 3n + 3 = 3(n+1) $$

  Por tanto el valor de la cota asintótica superior $g(n)$ con $n_0 = 10$ es:

  $$ g(n) = kn $$

  $$ g(n) > T(n) $$

  $$ kn > 3(n+1) $$
  $$ k > \frac{3(n+1)}{n} $$
  $$ k > 3 + \frac{3}{n} $$
  $$ n_0 = 10 $$
  $$ k > 3 + \frac{3}{10} $$
  $$ k > \frac{33}{10} $$

  $$ k > 3.3 $$
  $$ k = 3.31 $$

  $$ g(n) = 3.31n $$


  \subsection{MT No Determinista de 2 cintas}

  \subsubsection{Implementación Propuesta}

  \begin{figure}[!h]
    \imgcenter[150]{Images/ej0c1.png}
    \caption{MT No Determinista de 2 cintas - Palíndromos}
  \end{figure}


  \subsubsection{Determinación del Peor Caso}

  Dado que esta MT tan solo reconoce palíndromos, se toma como peor caso el único caso reconocido por la máquina: palíndromos de tamaño par.

  \subsubsection{Simulación con Diferentes Tamaños}

  \begin{table}[!h]
    \centering
  \begin{tabular}{|c|c|c|}
  \hline

  \textbf{Entrada} & \textbf{Tamaño} & \textbf{Pasos} \\ \hline

  $\lambda$           & 0               & 2              \\ \hline
  aa               & 2               & 3              \\ \hline
  abba             & 4               & 5              \\ \hline
  abaaba           & 6               & 7              \\ \hline
  ababbaba         & 8               & 9              \\ \hline
  ababaababa       & 10              & 11             \\ \hline
  \end{tabular}
  \caption{Diferentes tamaños}
  \end{table}



  \subsubsection{Cálculo de $T(n)$}


  \begin{table}[!h]
    \centering
  \begin{tabular}{|c|p{1cm}|p{1cm}|p{1cm}|p{1cm}|p{1cm}|p{1cm}|}
  \hline
  \textbf{N}     & \multicolumn{1}{c|}{\textbf{0}} & \multicolumn{1}{c|}{\textbf{2}} & \multicolumn{1}{c|}{\textbf{4}} & \multicolumn{1}{c|}{\textbf{6}} & \multicolumn{1}{c|}{\textbf{8}} & \multicolumn{1}{c|}{\textbf{10}} \\ \hline
  \textbf{Pasos} & \multicolumn{1}{c|}{2}          & \multicolumn{1}{c|}{3}          & \multicolumn{1}{c|}{5}         & \multicolumn{1}{c|}{7}         & \multicolumn{1}{c|}{9}         & \multicolumn{1}{c|}{11}          \\ \hline
  \textbf{Diferencia 1}                  &                                                         & 1                                                       & 2                               & 2                              & 2                              & 2                               \\ \hline
  \textbf{Diferencia 2}                  &                                                         & \multicolumn{1}{r|}{1}                                  & \multicolumn{1}{r|}{0}          & \multicolumn{1}{r|}{0}          & \multicolumn{1}{r|}{0}          &                                  \\ \hline
  \end{tabular}
  \caption{Diferencias finitas}
  \end{table}

    Nota: Para poder aceptar lambda es necesario utilizar una regla especializada, que hace que no se cumplan las diferencias finitas, añadiendo un paso más. Para el cálculo de T(n) se ignora este primer caso.

    \vspace{2mm}

    Dado que en la \textit{Diferencia 1} se encuentran valores constantes, es una ecuación de primer grado:


    $$ T(n) = an+ b $$

    Despejando sus valores se obtiene:

    $$ T(2) = 2a + b = 3 $$
    $$ T(4) = 4a + b = 5 $$

    $$ a = 1 , \quad b = 1$$

    La complejidad de esta máquina de Turing es:

    $$ T(n) = n + 1 \quad \forall \; n>0 $$

    Por tanto el valor de la cota asintótica superior $g(n)$ con $n_0 = 10$ es:

    $$ g(n) = kn $$

    $$ g(n) > T(n) $$

    $$ kn > n + 1 $$
    $$ k > 1 + \frac{1}{n} $$
    $$ n_0 = 10 $$
    $$ k > 1 + \frac{1}{10} $$
    $$ k > \frac{11}{10} $$

    $$ k > 1.1 $$
    $$ k = 1.11 $$

    $$ g(n) = 1.11n $$


  \newpage
  \section{Suma de enteros en base UNO}



  \subsection{MT Determinista de 1 cinta}

  \subsubsection{Implementación Propuesta}

  \begin{figure}[!h]
    \imgcenter[150]{Images/ej1a.png}
    \caption{MT Determinista de 1 cinta - Suma de enteros en base UNO}
  \end{figure}


  \subsubsection{Determinación del Peor Caso}

  En este problema el peor caso se encuentra cuando la parte izquierda de la suma está vacía y la parte derecha tiene todos los '1'. Esto se debe a que por cada '1' en la parte derecha, la máquina de Turing tiene que recorrer la tira entera hasta la izquierda.

  \begin{table}[!h]
    \centering
  \begin{tabular}{|c|c|c|}
  \hline
  \textbf{Entrada} & \textbf{Pasos} & \textbf{Resultado} \\ \hline
  \$111            & 37             & 111              \\ \hline
  1\$11            & 28             & 111              \\ \hline
  11\$1            & 19             & 111              \\ \hline
  111\$            & 10             & 111              \\ \hline
  \end{tabular}
  \caption{Peor caso}
  \end{table}


  \subsubsection{Simulación con Diferentes Tamaños}
  \begin{table}[!h]
    \centering
  \begin{tabular}{|c|c|c|}
  \hline
  \textbf{Entrada} & \textbf{Tamaño} & \textbf{Pasos} \\ \hline
  \$               & 1               & 4              \\ \hline
  \$1              & 2               & 11             \\ \hline
  \$11             & 3               & 22             \\ \hline
  \$111            & 4               & 37             \\ \hline
  \$1111           & 5               & 56             \\ \hline
  \end{tabular}
  \caption{Diferentes tamaños}
  \end{table}


  \subsubsection{Cálculo de $T(n)$}

  \begin{table}[!h]
    \centering
    \begin{tabular}{|c|p{1cm}|p{1cm}|p{1cm}|p{1cm}|p{1cm}|p{1cm}|}
  \hline
  \textbf{N}     & \multicolumn{1}{c|}{\textbf{1}} & \multicolumn{1}{c|}{\textbf{2}} & \multicolumn{1}{c|}{\textbf{3}} & \multicolumn{1}{c|}{\textbf{4}} & \multicolumn{1}{c|}{\textbf{5}} \\ \hline
  \textbf{Pasos} & \multicolumn{1}{c|}{4}          & \multicolumn{1}{c|}{11}         & \multicolumn{1}{c|}{22}         & \multicolumn{1}{c|}{37}         & \multicolumn{1}{c|}{56}         \\ \hline
  \textbf{Diferencia 1}                  &                                                         & 7                                                       & 11                              & 15                              & 19                              \\ \hline
  \textbf{Diferencia 2}                  &                                                         & \multicolumn{1}{r|}{4}                                  & \multicolumn{1}{r|}{4}          & \multicolumn{1}{r|}{4}          &                                 \\ \hline
  \textbf{Diferencia 3}                  &                                                         &                                                         & 0                               & 0                               &                                 \\ \hline
  \end{tabular}
  \caption{Diferencias finitas}
  \end{table}



  Dado que en la \textit{Diferencia 2} se encuentran valores constantes, es una ecuación de segundo grado:

  $$ T(n) = an^2 + bn + c $$

  Despejando sus valores se obtiene:

  $$ T(1) = a + b + c = 4 $$
  $$ T(2) = 4a + 2b + c = 11 $$
  $$ T(3) = 9a + 3b + c = 22 $$

  $$ a = 2 ,\quad b = 1 ,\quad c = 1 $$

  La complejidad de esta máquina de Turing es:

  $$ T(n) = 2n^2 + n + 1 $$

  Por tanto el valor de la cota asintótica superior $g(n)$ con $n_0 = 10$ es:

  $$ g(n) = kn^2 $$

  $$ g(n) > T(n) $$

  $$ kn^2 > 2n^2 + n + 1 $$
  $$ k > 2 + \frac{1}{n} + \frac{1}{n^2} $$
  $$ n_0 = 10 $$
  $$ k > 2 + \frac{1}{10} + \frac{1}{10^2} $$
  $$ k > 2 + \frac{1}{10} + \frac{1}{100} $$
  $$ k > \frac{211}{100} $$

  $$ k > 2.11 $$
  $$ k = 2.12 $$

  $$ g(n) = 2.12n^2 $$







  \newpage

  \subsection{MT Determinista de 2 cintas}

  \subsubsection{Implementación Propuesta}


  \begin{figure}[!h]
    \imgcenter[150]{Images/ej1b.png}
    \caption{MT Determinista de 2 cintas - Suma de enteros en base UNO}
  \end{figure}


  \subsubsection{Determinación del Peor Caso}

  Al igual que en la versión con una sola cinta, el peor caso surge cuando la parte izquierda se encuentra vacía, y la parte derecha tiene todos los '1'.

  \begin{table}[!h]
    \centering
  \begin{tabular}{|c|c|c|}
  \hline
  \textbf{Entrada} & \textbf{Pasos} & \textbf{Resultado} \\ \hline
  \$111            & 14             & 111                \\ \hline
  1\$11            & 13             & 111                \\ \hline
  11\$1            & 12             & 111                \\ \hline
  111\$            & 11             & 111                \\ \hline
  \end{tabular}
  \caption{Peor caso}
  \end{table}

  \subsubsection{Simulación con Diferentes Tamaños}

  \begin{table}[!h]
    \centering
  \begin{tabular}{|c|c|c|}
  \hline

  \textbf{Entrada} & \textbf{Tamaño} & \textbf{Pasos} \\ \hline

  \$      & 1               & 5              \\ \hline
  \$1     & 2               & 8              \\ \hline
  \$11    & 3               & 11             \\ \hline
  \$111   & 4               & 14             \\ \hline
  \$1111           & 5               & 17             \\ \hline
  \end{tabular}
  \caption{Diferentes tamaños}
  \end{table}



  \subsubsection{Cálculo de $T(n)$}

  \begin{table}[!h]
    \centering
  \begin{tabular}{|c|p{1cm}|p{1cm}|p{1cm}|p{1cm}|p{1cm}|p{1cm}|}
  \hline
  \textbf{N}     & \multicolumn{1}{c|}{\textbf{1}} & \multicolumn{1}{c|}{\textbf{2}} & \multicolumn{1}{c|}{\textbf{3}} & \multicolumn{1}{c|}{\textbf{4}} & \multicolumn{1}{c|}{\textbf{5}} \\ \hline
  \textbf{Pasos} & \multicolumn{1}{c|}{5}          & \multicolumn{1}{c|}{8}          & \multicolumn{1}{c|}{11}         & \multicolumn{1}{c|}{14}         & \multicolumn{1}{c|}{17} \\ \hline
  \textbf{Diferencia 1}                  &                                                         & 3                                                       & 3                               & 3                              & 3                             \\ \hline
  \textbf{Diferencia 2}                  &                                                         & \multicolumn{1}{r|}{0}                                  & \multicolumn{1}{r|}{0}          & \multicolumn{1}{r|}{0}       &                         \\ \hline
  \end{tabular}
  \caption{Diferencias finitas}
  \end{table}

  Dado que en la \textit{Diferencia 1} se encuentran valores constantes, es una ecuación de primer grado:


  $$ T(n) = an + b $$

  Despejando sus valores se obtiene:

  $$ T(1) = a + b = 5 $$
  $$ T(2) = 2a + b = 8 $$

  $$ a = 3 , \quad b = 2$$

  La complejidad de esta máquina de Turing es:

  $$ T(n) = 3n + 2$$

  Por tanto el valor de la cota asintótica superior $g(n)$ con $n_0 = 10$ es:

  $$ g(n) = kn $$

  $$ g(n) > T(n) $$

  $$ kn > 3n + 2 $$
  $$ k > 3 + \frac{2}{n} $$
  $$ n_0 = 10 $$
  $$ k > 3 + \frac{2}{10} $$
  $$ k > \frac{32}{10} $$

  $$ k > 3.2 $$
  $$ k = 3.21 $$

  $$ g(n) = 3.21n $$



  \subsection{Evaluación de la mejora obtenida con la MT de 2 cintas}


  \begin{figure}[!h]
    \imgcenter[170]{Images/comparativaBaseUno.png}
    \caption{Comparativa MT 1 cinta vs. MT 2 cintas}
    \label{fig:comparativa1}
  \end{figure}

  La Máquina de Turing de suma de números en base UNO con una sola cinta tiene un coste polinomial de grado 2 ($T(n) = 2n^2 + n + 1$), mientras que la que tiene dos cintas es tan solo de grado 1 ($T(n) = 3n + 2$), de forma que la máquina que tiene dos cintas tiene un menor coste computacional. Empíricamente se puede ver con la figura \ref{fig:comparativa1} la diferencia entre estas dos funciones.


  \vspace{4mm}

  Esto se debe a que la MT de una cinta tiene que recorrer la palabra entera de izquierda a derecha continuamente, mientras que la MT que usa dos cintas le basta con que cada una recorra uno de los dos operandos.



  \clearpage


  \section{Suma de enteros en base DOS}



  \subsection{MT Determinista de 1 cinta}

  \subsubsection{Implementación Propuesta}

  \begin{figure}[!h]
    \imgcenter[150]{Images/ej2a.png}
    \caption{MT Determinista de 1 cinta - Suma de enteros en base DOS}
  \end{figure}


  \subsubsection{Determinación del Peor Caso}

  En este problema, el peor caso se encuentra cuanto más grande es el segundo número y el primer número está vacío, de forma a similar a lo que ocurría en el caso de la suma en base UNO.

  \begin{table}[!h]
    \centering
  \begin{tabular}{|c|c|c|}
  \hline
  \textbf{Entrada} & \textbf{Pasos} & \textbf{Resultado} \\ \hline
  \$00             & 14             & cinta vacía        \\ \hline
  \$01             & 23             & 1                  \\ \hline
  \$10             & 34             & 10                 \\ \hline
  \$11             & 42             & 11                 \\ \hline
  0\$0             & 12             & 0                  \\ \hline
  0\$1             & 12             & 1                  \\ \hline
  1\$0             & 18             & 1                  \\ \hline
  1\$1             & 21             & 10                 \\ \hline
  00\$             & error          & error              \\ \hline
  01\$             & error          & error              \\ \hline
  10\$             & error          & error              \\ \hline
  11\$             & error          & error              \\ \hline
  1\$11            & 48             & 100                \\ \hline
  11\$1            & 25             & 100                \\ \hline
  \$111            & 99             & 111                \\ \hline
  0\$111           & 99             & 111                \\ \hline
  \end{tabular}
  \caption{Peor caso}
  \end{table}


  \subsubsection{Simulación con Diferentes Tamaños}

  \begin{table}[!h]
    \centering
  \begin{tabular}{|c|c|c|}
  \hline
  \textbf{Entrada} & \textbf{Tamaño} & \textbf{Pasos} \\ \hline
  \$1              & 2               & 17             \\ \hline
  \$11             & 3               & 42             \\ \hline
  \$111            & 4               & 99             \\ \hline
  \$1111           & 5               & 228            \\ \hline
  \$11111          & 6               & 517            \\ \hline
  \end{tabular}
  \caption{Diferentes tamaños}
  \end{table}



  \subsubsection{Cálculo de $T(n)$}

  \begin{table}[!h]
    \centering
  \begin{tabular}{|c|p{1cm}|p{1cm}|p{1cm}|p{1cm}|p{1cm}|p{1cm}|}
  \hline
  \textbf{N}     & \multicolumn{1}{c|}{\textbf{2}} & \multicolumn{1}{c|}{\textbf{3}} & \multicolumn{1}{c|}{\textbf{4}} & \multicolumn{1}{c|}{\textbf{5}} & \multicolumn{1}{c|}{\textbf{6}} \\ \hline
  \textbf{Pasos} & \multicolumn{1}{c|}{17}          & \multicolumn{1}{c|}{42}          & \multicolumn{1}{c|}{99}         & \multicolumn{1}{c|}{228}         & \multicolumn{1}{c|}{517} \\ \hline
  \textbf{Diferencia 1}                  &                                                         & 25                                                       & 57                               & 129                              & 289                             \\ \hline
  \textbf{Diferencia 2}                  &                                                         & \multicolumn{1}{r|}{32}                                  & \multicolumn{1}{r|}{72}          & \multicolumn{1}{r|}{160}       &                         \\ \hline
  \end{tabular}
  \caption{Diferencias finitas}
  \end{table}


  El calculo de las diferencias finitas es inconcluyente dado que no se consigue apreciar ningún patron representativo. Por ese motivo se plantea realizar un estudio analítico sobre esta máquina.

  \vspace{2mm}

  La MT está compuesta por un bucle inicial y una parte adicional que limpia la cinta y mantiene el resultado final. El bucle a su vez está compuesto por un primer paso que requiere de ir desde la izquierda del todo de la cinta a la parte derecha. Después se resta '1' siguiendo la aritmética de Peano, realizando acarreos cuando es pertinente. A continuación se queda en el último dígito del primer término, y realiza una suma, nuevamente con accarreo cuando es necesario.

  \vspace{2mm}

  La primera vez que se va hacia la derecha, tiene un coste de $n$. Posteriormente, se vuelve hasta la izquierda del todo al hacer la resta del segundo término, con un coste nuevamente de $n$. La suma en el primer término es un cálculo variable, e inicialmente se desprecia. El bucle se tiene que realizar tantas veces como valor tenga en binario, y dado que este término está compuesto por '1's únicamente, y tiene un tamaño de $n-1$, el bucle se realiza $2^{n-1}-1$ veces.

  \vspace{2mm}

  Por otro lado, la parte final del algoritmo requiere de ir hasta la derecha del todo nuevamente ($n$), volver hacia a la izquierda acarreando en la resta ($n$), una vez más va hacia la derecha para ponerse en la parte final ($n$), se vuelve borrando hacia la izquierda para eliminar el segundo término ($n$) y se recorre hacia la izquierda hasta ponerse en el inicio del resultado, quie tiene un tamaño $n$ dado que tiene que dar lo mismo que la entrada inicial.

  \vspace{2mm}


  Analizando la complejidad de la máquina, se concluye inicialmente que es:

  $$5n + ((2^{n-1}) - 1) * 2n$$



    Sin embargo, a la hora de aplicar la función anterior, queda lo siguiente en pasos esperados:


    \begin{table}[!h]
      \centering
    \begin{tabular}{|c|c|c|c|c|}
    \hline
    \textbf{Entrada} & \textbf{Tamaño} & \textbf{Pasos Obtenido} & \textbf{Pasos Esperado} & \textbf{Resta}\\ \hline
    \$1              & 2               & 17                      & 14                        &  3             \\ \hline
    \$11             & 3               & 42                      & 33                        &  9             \\ \hline
    \$111            & 4               & 99                      & 76                        &  23             \\ \hline
    \$1111           & 5               & 228                     & 175                       &  53           \\ \hline
    \$11111          & 6               & 517                     & 402                       &  115         \\ \hline
    \end{tabular}
    \caption{Cálculo de complejidad $T(n)$ con estudio analítico}
    \end{table}


    Para comprobar que es lo que le falta a esta función, se realizan diferencias finitas sobre la resta entre lo obtenido y lo esperado.



    \begin{table}[!h]
      \centering
    \begin{tabular}{|c|p{1cm}|p{1cm}|p{1cm}|p{1cm}|p{1cm}|p{1cm}|}
    \hline
    \textbf{N}     & \multicolumn{1}{c|}{\textbf{2}} & \multicolumn{1}{c|}{\textbf{3}} & \multicolumn{1}{c|}{\textbf{4}} & \multicolumn{1}{c|}{\textbf{5}} & \multicolumn{1}{c|}{\textbf{6}} \\ \hline
    \textbf{Resta} & \multicolumn{1}{c|}{3}& \multicolumn{1}{c|}{9}& \multicolumn{1}{c|}{23}& \multicolumn{1}{c|}{53}& \multicolumn{1}{c|}{115} \\ \hline
    \textbf{Diferencia 1}& & 6& 14& 30& 62\\ \hline
    \textbf{Diferencia 2} & &\multicolumn{1}{r|}{8} &\multicolumn{1}{r|}{16} &\multicolumn{1}{r|}{32} & \\ \hline
    \end{tabular}
    \caption{Diferencias finitas}
    \end{table}

    Se obtiene una diferencia que coindice con un crecimiento exponencial (potencia perfecta de $2^n$ en la Diferencia 2), lo que significa que aún se puede ajustar la complejidad un poco más. En este caso, la nueva función quedaría como $5n + ((2^{n-1}) - 1) * 2n + 2^{n+1}$, ya que para $n = 1$ no se aplica esta complejidad (por el hecho de que un input de tamaño 1 es simplemente un \$ y no hay operaciones que hacer).

    \vspace{2mm}

    Aún así hay un pequeño desajuste entre la función obtenida actualmente y los valores obtenidos empíricamente:

    \begin{table}[!h]
      \centering
    \begin{tabular}{|c|c|c|c|c|}
    \hline
    \textbf{Entrada} & \textbf{Tamaño} & \textbf{Pasos Obtenido} & \textbf{Pasos Esperado} & \textbf{Resta}\\ \hline
    \$1              & 2               & 17                      & 22                        & -5             \\ \hline
    \$11             & 3               & 42                      & 49                        &  -7             \\ \hline
    \$111            & 4               & 99                      & 108                        &  -9             \\ \hline
    \$1111           & 5               & 228                     & 239                       &  -11           \\ \hline
    \$11111          & 6               & 517                     & 530                       &  -13         \\ \hline
    \end{tabular}
    \caption{Cálculo de complejidad $T(n)$ con estudio analítico y exponencial}
    \end{table}

    Por tanto, se vuelvenm a realizar diferencias finitas sobre la resta.

    \begin{table}[!h]
      \centering
    \begin{tabular}{|c|p{1cm}|p{1cm}|p{1cm}|p{1cm}|p{1cm}|p{1cm}|}
    \hline
    \textbf{N}     & \multicolumn{1}{c|}{\textbf{2}} & \multicolumn{1}{c|}{\textbf{3}} & \multicolumn{1}{c|}{\textbf{4}} & \multicolumn{1}{c|}{\textbf{5}} & \multicolumn{1}{c|}{\textbf{6}} \\ \hline
    \textbf{Resta} & \multicolumn{1}{c|}{-5}& \multicolumn{1}{c|}{-7}& \multicolumn{1}{c|}{-9}& \multicolumn{1}{c|}{-11}& \multicolumn{1}{c|}{-13} \\ \hline
    \textbf{Diferencia 1}& & -2& -2& -2& -2\\ \hline
    \textbf{Diferencia 2} & &\multicolumn{1}{r|}{0} &\multicolumn{1}{r|}{0} &\multicolumn{1}{r|}{0} & \\ \hline
    \end{tabular}
    \caption{Diferencias finitas}
    \end{table}


    Ahora sí se ha llegado a una diferencia finita con un crecimiento polinómico. Por lo tanto se tiene que la complejidad ahora es:
    $$T(n) = an + b$$

    $$T(2) = 2n + b = -5$$
    $$T(3) = 3a + b = -7$$


    $$ a = -2 ;\quad b = -1$$

    Por tanto, la complejidad final es:

    $$ T(n) = 5n + ((2^{n-1}) - 1) * 2n + 2^{n+1} - 2n -1$$


    que comparando con los valores obtenidos empíricamente, resultan ser los mismos pasos, por lo que se puede decir que la complejidad final simplificando sale:

    $$T(n) = 3n + ((2^{n-1}) - 1) * 2n + 2^{n+1} - 1$$
    $$T(n) = 3n + (\frac{2^n}{2} - 1) * 2n + 2*2^n - 1$$
    $$T(n) = 3n + 2n*\frac{2^n}{2} - 2n + 2*2^n - 1$$
    $$T(n) = 3n + n*2^n - 2n + 2*2^n - 1$$
    $$T(n) = n + n*2^n + 2*2^n - 1$$



    $$T(n) = 2^n*(n + 2) + n - 1$$

    Por tanto, $g(n)$ tiene una forma $g(n) = k*n*2^n$

    $$g(n) > T(n)$$

    $$k*n*2^n > 2^n * (n + 2) + n - 1$$
    $$k > \frac{n + 2}{n} + \frac{1}{2^n} - \frac{1}{n*2^n}$$

    y con $n_0 = 10$

    $$k > \frac{10 + 2}{10} + \frac{1}{2^{10}} - \frac{1}{10*2^{10}}$$
    $$k > \frac{12}{10} + \frac{1}{1024} - \frac{1}{10240}$$
    $$k > 1.20087$$

    $$ k = 1.21$$

    $$g(n) = 1.21*n*2^n$$








  \subsection{MT Determinista de 2 cintas}

  \subsubsection{Implementación Propuesta}

  \begin{figure}[!h]
    \imgcenter[150]{Images/ej2b.png}
    \caption{MT Determinista de 2 cintas - Suma de enteros en base DOS}
  \end{figure}

  \newpage

  \subsubsection{Determinación del Peor Caso}

  En esta MT el peor caso vuelve a ser cuando el término de la izquierda esta vacío y el término de la derecha está compuesto únicamente por '1'.

  \begin{table}[!h]
    \centering
  \begin{tabular}{|c|c|c|}
  \hline
  \textbf{Entrada} & \textbf{Pasos} & \textbf{Resultado} \\ \hline
  \$00             & 10             & 00                 \\ \hline
  \$01             & 10             & 01                 \\ \hline
  \$10             & 10             & 10                 \\ \hline
  \$11             & 10             & 11                 \\ \hline
  0\$0             & 8              & 0                  \\ \hline
  0\$1             & 8              & 1                  \\ \hline
  1\$0             & 8              & 1                  \\ \hline
  1\$1             & 11             & 10                 \\ \hline
  11\$11           & 17             & 110                \\ \hline
  11\$10           & 15             & 101                \\ \hline
  111\$111         & 23             & 1110               \\ \hline
  11\$1111         & 27             & 10010              \\ \hline
  1\$11111         & 31             & 100000             \\ \hline
  1111\$11         & 25             & 10010              \\ \hline
  \end{tabular}
  \caption{Peor caso}
  \end{table}



  \subsubsection{Simulación con Diferentes Tamaños}

  \begin{table}[!h]
    \centering
  \begin{tabular}{|c|c|c|}
  \hline
  \textbf{Entrada} & \textbf{Tamaño} & \textbf{Pasos} \\ \hline
  \$     & 1               & 6             \\ \hline
  \$1    & 2               & 11             \\ \hline
  \$11   & 3               & 28             \\ \hline
  \$111  & 4               & 69             \\ \hline
  \$1111 & 5               & 166             \\ \hline
  \$11111         & 7               & 391             \\ \hline
  \end{tabular}
  \caption{Diferentes tamaños}
  \end{table}


  \subsubsection{Cálculo de $T(n)$}


  \begin{table}[!h]
    \centering
  \begin{tabular}{|c|p{1cm}|p{1cm}|p{1cm}|p{1cm}|p{1cm}|p{1cm}|p{1cm}|}
  \hline
  \textbf{N}     & \multicolumn{1}{c|}{\textbf{1}} & \multicolumn{1}{c|}{\textbf{2}} & \multicolumn{1}{c|}{\textbf{3}} & \multicolumn{1}{c|}{\textbf{4}} & \multicolumn{1}{c|}{\textbf{5}} & \multicolumn{1}{c|}{\textbf{7}}\\ \hline
  \textbf{Pasos} & \multicolumn{1}{c|}{6}          & \multicolumn{1}{c|}{11}          & \multicolumn{1}{c|}{28}         & \multicolumn{1}{c|}{69}         & \multicolumn{1}{c|}{166} & \multicolumn{1}{c|}{391} \\ \hline
  \textbf{Diferencia 1}& & 5 & 17 & 41 & 97 & 225\\ \hline
  \textbf{Diferencia 2}& & \multicolumn{1}{r|}{12}& \multicolumn{1}{r|}{24}& \multicolumn{1}{r|}{56} & \multicolumn{1}{r|}{128} & \\ \hline
  \end{tabular}
  \caption{Diferencias finitas}
  \end{table}

  En la tabla de diferencias finitas no se encuentra ni valor constante ni exponenciales perfectas por lo que se decide realizar un estudio analítico de la máquina.

  \vspace{2mm}

  Inicialmente la máquina recorre toda la cinta hasta la derecha del todo, copiando el segundo término en la cinta, lo cual tiene una complejidad de $n$. A continuación se borra el segundo término de la cinta 1, con un coste de $n-1$.

  \vspace{2mm}

  A continuación comienza el bucle de la aritméticas de Peano. Este bucle se realiza, al igual que para el caso anterior $ (2^{n-1} -1) $ veces. Dentro de este bucle es necesario ir hasta la izquierda completamente, con coste $n$, y de vuelta hacia la derecha, con otro coste $n$, porque, aunque no haya acarreo, siempre se va hasta el final de la palabra. Esto, a pesar de ser ineficiente, se hace por sinmplificar el problema y facilitar su análisis. Por tanto, la complejidad del bucle es:

  $$ (2^{n-1} -1)*2n $$


  Pero al compararse con los valores obtenidos, se puede apreciar que los valores no salen exactos usando esta función.

  \begin{table}[!h]
    \centering
  \begin{tabular}{|c|c|c|c|c|}
  \hline
  \textbf{Entrada} & \textbf{Tamaño} & \textbf{Pasos Obtenido} & \textbf{Pasos Esperado} & \textbf{Resta}\\ \hline
  \$              & 1               & 6                      & 0                        &  6             \\ \hline
  \$1             & 2               & 11                      & 4                        &  7             \\ \hline
  \$11            & 3               & 28                      & 18                        &  10             \\ \hline
  \$111           & 4               & 69                     & 56                       &  13          \\ \hline
  \$1111          & 5               & 166                     & 150                       &  16        \\ \hline
  \$11111          & 6               & 391                     & 372                       &  19        \\ \hline
  \end{tabular}
  \caption{Cálculo de complejidad $T(n)$ con estudio analítico}
  \end{table}


  Por tanto se realiza las diferencias finitas sobre la diferencia entre lo esperado y lo obtenido.




  \begin{table}[!h]
    \centering
  \begin{tabular}{|c|p{1cm}|p{1cm}|p{1cm}|p{1cm}|p{1cm}|p{1cm}|}
  \hline
  \textbf{N}     & \multicolumn{1}{c|}{\textbf{1}} & \multicolumn{1}{c|}{\textbf{2}} & \multicolumn{1}{c|}{\textbf{3}} & \multicolumn{1}{c|}{\textbf{4}} & \multicolumn{1}{c|}{\textbf{5}} & \multicolumn{1}{c|}{\textbf{6}} \\ \hline
  \textbf{Resta} & \multicolumn{1}{c|}{6}& \multicolumn{1}{c|}{7}& \multicolumn{1}{c|}{10}& \multicolumn{1}{c|}{13}& \multicolumn{1}{c|}{16} & \multicolumn{1}{c|}{19}\\ \hline
  \textbf{Diferencia 1}& & 1& 3& 3& 3& 3\\ \hline
  \textbf{Diferencia 2}& & \multicolumn{1}{r|}{2}& \multicolumn{1}{r|}{0}& \multicolumn{1}{r|}{0} & \multicolumn{1}{r|}{0} & \\ \hline
  \end{tabular}
  \caption{Diferencias finitas}
  \end{table}

  Se puede apreciar que en la diferencia 1 se obtiene un valor constante, por tanto se puede decir que $T(n)$ es:

  $$ T'(n)=an + b $$

  $$ T'(2) = 2a + b = 7$$
  $$ T'(3) = 3a + b = 10$$

  $$ a = 3 ;\quad b = 1 $$

  $$ T'(n) = 3n + 1 $$

  Por tanto el $T(n)$ total es:

  $$ T(n) = (2^{n-1} - 1) * 2n + 3n + 1 $$

  y simplificando:

  $$ T(n) = 2n*\frac{2^n}{2} - 2n + 3n + 1 $$
  $$ T(n) = n* 2^n + n + 1 $$
  $$ T(n) = n*(2^n + 1) + 1 $$

  \vspace{2mm}

  $$ g(n) = k*n*2^n $$

  $$ g(n) > T(n) $$
  $$ k*n*2^n > n*(2^n + 1) + 1 $$
  $$ k > \frac{2^n + 1}{2^n} + \frac{1}{n*2^n} $$

  y con $n_0 = 10$


  $$ k > \frac{2^{10} + 1}{2^{10}} + \frac{1}{10*2^{10}} $$
  $$ k > \frac{1025}{1024} + \frac{1}{10240} $$
  $$ k > 1.00107 $$

  $$ k = 1.01 $$

  $$g(n) = 1.01 * n * 2^n$$




  \subsection{Evaluación de la mejora obtenida con la MT de 2 cintas}

  Ambas máquinas comparten una complejidad similar de $O(n) = n*2^n$. Muy probablemente la complejidad de la máquina con 2 cintas podría reducirse a una complejidad $O(n) = 2^n$ o similar, pero, tal y como se ha mencionado, con el objetivo de simplificar y facilitar su análisis, se opta por un diseño menos óptimo.

  \vspace{2mm}

  Por este motivo, tienen una complejidad muy similar, tal y como se puede ver en la gráfica.

  \begin{figure}[!h]
    \imgcenter[160]{Images/comparativaDOS.png}
    \caption{Comparativa MT 1 cinta vs. MT 2 cintas}
  \end{figure}


  \clearpage

  \section{Comparativa de las MT de suma de base UNO y suma de base DOS}

  \begin{table}[!h]
    \centering
  \begin{tabular}{|c|c|c|c|}
  \hline
  \textbf{Base 1}   & \textbf{Base 2} & \textbf{Pasos BASE 1} & \textbf{Pasos BASE 2} \\ \hline
  1\$1     & 1\$1            & 10                    & 13                    \\ \hline
  11\$11   & 10\$10          & 15                    & 25                    \\ \hline
  111\$111 & 11\$11          & 20                    & 35                    \\ \hline
  1111\$1111        & 100\$100        & 25                    & 49                    \\ \hline
  11111\$11111      & 101\$101        & 30                    & 61                    \\ \hline
  111111\$111111    & 110\$110        & 35                    & 73                    \\ \hline
  1111111\$1111111  & 111\$111        & 40                    & 85                    \\ \hline
  \$1               & \$1             & 8                     & 11                    \\ \hline
  \$11              & \$10            & 11                    & 22                    \\ \hline
  \$111             & \$11            & 14                    & 28                    \\ \hline
  \$1111            & \$100           & 17                    & 45                    \\ \hline
  \$11111           & \$101           & 20                    & 53                    \\ \hline
  \$111111          & \$110           & 23                    & 61                    \\ \hline
  \$1111111         & \$111           & 26                    & 69                    \\ \hline
  \end{tabular}
  \caption{Comparativa pasos MT 2 cintas suma base 1 y base 2}
  \label{tab:comparativaSuma}
  \end{table}


  Como se aprecia en la tabla \ref{tab:comparativaSuma}, el número de pasos de la suma en base 2 es mayor que los obtenidos con base 1 para todos los casos. En la figura \ref{fig:comparativaEJ1y2} también se puede apreciar como las complejidades difieren en gran medida. Para esta gráfica, dado que la fórmula obtenida previamente no es aplicable a todos los ejemplos en los que se usa, se realiza una aproximación con el objetivo de representar visualmete la diferencia de complejidades.

  \vspace{2mm}

  Las entradas que se tienen en cuenta en la gráfica \ref{fig:comparativaEJ1y2} son de la forma $\$x$ donde $x$ es el número de entrada, ya sea en base UNO o base DOS. El eje x de la gráfica hace referencia a este número, mostrándolo en base 10.



  \begin{figure}[!h]
    \imgcenter[130]{Images/comparativaEJ1y2.png}
    \caption{Gráfica comparativa pasos MT 2 cintas suma base 1 y base 2}
    \label{fig:comparativaEJ1y2}
  \end{figure}


  Esta diferencia de complejidades se debe al acarreo que tiene que tener en cuenta la MT en la suma en base 2, ya que esto hace que aumente considerablemente el número de pasos requerido para realizar la suma. Las complejidades de cada algoritmo respecto a $x$ son:

  $$ T_{UNO}(x) = 3x + 5 $$

  $$ T_{DOS}(x) = (2^{\log_2(x)+1} - 1) * 2*(\log_2(x))+ 3*(\log_2(x)) + 11  $$
  $$ T_{DOS}(x) = (2x - 1) * 2*\log_2(x) + 3*\log_2(x) + 11  $$


  Por lo tanto se puede ver que la complejidad de la suma en base 1 es lineal y la de la suma en base 2 es lineal logarítmica, que es peor computacionalmente hablando.



  \clearpage


  \section{Palíndromo de Orden k}

  \subsection{MT Determinista de 3 cintas}


  \subsubsection{Implementación Propuesta}

  La máquina que se plantea cuenta con 3 cintas. Dada una entrada de la forma $ k\$x $, donde $k$ es el orden del palíndromo y $x$ es la palabra, el funcionamiento general es:
    \begin{enumerate}
      \item Se recorre la cinta 1 hasta que se encuentre blanco, pero a partir de que se encuentre el símbolo $\$$ copia en la cinta 2 todo lo que hay en la 1, es decir, la palabra.
      \item Se rebobinan ambas cintas, en la cinta 1 se va borrando hasta llegar a $k$, y en la cinta 2 se rebobina hasta ponerse en el símbolo de más a la izquierda sin modificar nada.
      \item Se entra en un bucle para determinar si la palabra es palíndromo de orden $k$:
      \begin{enumerate}
        \item Se borra el primer símbolo de la izquierda de la palabra de la cinta 2.
        \item Se recorre la cinta 2 entera hasta la derecha del todo.
        \item Si el símbolo coincide con el borrado al principio, se apunta en la cinta 3 el símbolo, y se borra en la cinta 2. Esto se hace para almacenar la mitad de la palabra en caso de que sea palíndromo, para poder calcular el siguiente orden.
        \item Si el símbolo no coincide, se termina, dado que se puede concluir que la palabra no es palíndromo.
        \item Esto se realiza hasta que la cinta 2 se encuentra vacía. Entonces se elimina un '1' de la cinta 1, y se copia en la cinta 2 el contenido de la cinta 3, mientras se borra esta tercera cinta.
      \end{enumerate}
      \item El bucle termina cuando se ha demostrado que no es palíndromo del orden requerido, o cuando la cinta 1 no tiene ningún '1' más, por lo que se cumple la condición de que $x$ es al menos palíndromo de orden $k$, y por tanto termina el algoritmo.
    \end{enumerate}

    \begin{figure}[!h]
      \imgcenter[175]{Images/ej4a.png}
      \caption{MT Determinista de 3 cintas - Comprobación de polinomios de grado $k$}
    \end{figure}


  \subsubsection{Determinación del Peor Caso}

  \begin{table}[!h]
    \centering
  \begin{tabular}{|c|c|c|}
  \hline
  \textbf{Entrada} & \textbf{Pasos} & \textbf{Resultado} \\ \hline
  \$aaa   & 13             & 0                  \\ \hline
    \$aab   & 13             & 0                  \\ \hline
  1\$aa   & 16             & 1                  \\ \hline
  1\$ab            & 15             & 0                  \\ \hline
  11\$a            & 14             & 0                  \\ \hline
  111\$            & 10             & 0                  \\ \hline
  \end{tabular}
  \caption{Peor caso}
  \end{table}

  El peor caso se da cuando la palabra $x$ es palíndromo de orden $k$ porque es el caso en el que tiene que recorrer más veces el bucle. $x$ tendrá un tamaño igual a $2^k$ porque es el tamaño del palíndromo más pequeño que existe de orden $k$ y está compuesto por el mismo símbolo exclusivamente.


  \subsubsection{Simulación con Diferentes Tamaños}

  \begin{table}[!h]
    \centering
  \begin{tabular}{|c|c|c|}
  \hline
  \textbf{Entrada}       & \textbf{Tamaño} & \textbf{Pasos} \\ \hline
  \$            & 1               & 4              \\ \hline
  1\$aa         & 4               & 16             \\ \hline
  11\$aaaa      & 7               & 39             \\ \hline
  111\$aaaaaaaa          & 12              & 98             \\ \hline
  1111\$aaaaaaaaaaaaaaaa & 21              & 277            \\ \hline
  \end{tabular}
  \caption{Diferentes tamaños}
  \end{table}





  \subsubsection{Cálculo de $T(n)$}


  \begin{table}[!h]
    \centering
  \begin{tabular}{|c|p{1cm}|p{1cm}|p{1cm}|p{1cm}|p{1cm}|p{1cm}|}
  \hline
  \textbf{N}     & \multicolumn{1}{c|}{\textbf{1}} & \multicolumn{1}{c|}{\textbf{4}} & \multicolumn{1}{c|}{\textbf{7}} & \multicolumn{1}{c|}{\textbf{12}} & \multicolumn{1}{c|}{\textbf{21}} \\ \hline
  \textbf{Pasos} & \multicolumn{1}{c|}{4}          & \multicolumn{1}{c|}{16}          & \multicolumn{1}{c|}{39}         & \multicolumn{1}{c|}{98}         & \multicolumn{1}{c|}{277} \\ \hline
  \textbf{Diferencia 1}                  &                                                         & 12                                                       & 23                               & 59                              &   179                           \\ \hline
  \textbf{Diferencia 2}                  &                                                         & \multicolumn{1}{r|}{11}                                  & \multicolumn{1}{r|}{36}          & \multicolumn{1}{r|}{120}       &                         \\ \hline
  \end{tabular}
  \caption{Diferencias finitas}
  \label{tab:diferencias_finitas_pk}
  \end{table}




  Como se puede ver en la tabla \ref{tab:diferencias_finitas_pk}, por el método de diferencias finitas no se consiguen sacar valores constantes y por tanto se realiza otro tipo de análisis. Se analiza el comportamiento de la máquina para poder encontrar su complejidad.

  \vspace{2mm}

  La MT planteada se puede dividir en 2 bloques diferenciados. La preparación de las cintas y el bucle que comprueba la condición. Como se ha visto en apartados anteriores, el peor caso se da cuando $x$ es igual a $2^k$ porque esto quiere decir que $x$ es el palíndromo de orden $k$ más pequeño.

  \vspace{2mm}


  La preparación de la cinta depende del tamaño de la misma. Primero se recorre completamente hacia la derecha, hasta que se encuentra blanco, y después se vuelve hasta $k$, por tanto se recorre $k + 1(\$) + x + 1(blanco) + x + 1(\$)$ y dado que $x=2^k$, la complejidad de la preparación de las cintas es:

  $$ T_{prep}(k) = k + 2*2^k + 3 $$

  \vspace{2mm}




  Por otro lado, para el bucle se observa que tiene una especie de comportamiento recursivo. Siendo el caso base $ 1\$aa $, se aprecia que el siguiente palíndromo ($11\$aaaa $) contiene a este. El siguiente contiene a su vez a estos 2 últimos y así sucesivamente. Esto se puede ver en la tabla \ref{tab:detalle} donde el resultado total de cada ejemplo es:


  $$ T_{bucle}(k) = T_{bucle}(k-1) + y $$




  Donde $y$ representa el coste de comprobar si $x$ es palíndromo o no. Este coste es determinado por su longitud. Dado que cada vez se tiene que recorrer menos distancia de derecha a izquierda, porque cada vez se va acortando la palabra según se van haciendo comprobaciones, resulta en un proceso iterativo. A continuación se muestra un ejemplo de este funcionamiento:

  $$ Cinta_1: 1\mathbf{1} $$
  $$ Cinta_2: \mathbf{a}bba $$
  $$ Cinta_3:  $$


  $$ Cinta_1: 1\mathbf{1} $$
  $$ Cinta_2: bb\mathbf{a} $$
  $$ Cinta_3:  $$


  $$ Cinta_1: 1\mathbf{1} $$
  $$ Cinta_2: \mathbf{b}b $$
  $$ Cinta_3: \mathbf{a} $$


  $$ Cinta_1: 1\mathbf{1} $$
  $$ Cinta_2: \mathbf{b} $$
  $$ Cinta_3: \mathbf{a} $$

  $$ Cinta_1: 1\mathbf{1} $$
  $$ Cinta_2: $$
  $$ Cinta_3: a\mathbf{b} $$

  Este proceso está seguido por un copiado de la cinta 3 en la cinta 2 de nuevo, y la eliminación de un '1' de la cinta 1, borrando en el proceso la cinta 3, para poder repetir comprobación del palíndromo, pero con un tamaño reducido. Esto tiene un coste de $\frac{x}{2} + 2$, porque ahora el tamaño de $x$ es la mitad, y añade 1 paso más cada vez que se llega al final o principio de la palabra, porque se encuentra un blanco y debe cambiar de dirección.

  \vspace{2mm}

  Por tanto, la complejidad de realizar el bucle es:

  $$ y = \frac{x}{2} + 2 + x + \displaystyle\sum_{i=1}^x i $$

  lo que en función de $k$ se esxpresa como:

  $$ y = 2 + \frac{3*2^k}{2} + \displaystyle\sum_{i=1}^{2^k} i $$


  El sumatorio que se añade representa el número de pasos que se realizan cuando se desplaza el cabezal de izquierda a derecha según se van borrando los extremos de la palabra. Adicionalmente, dado que el paso se hace $x$ veces, se añade este valor porque es el número de veces que encuentra blanco y tiene que cambiar de dirección, añadiendo un paso más cada vez.



  \vspace{2mm}

  La complejidad total de esta máquina de Turing es:


  $$ T(k) = T_{prep}(k) + T_{bucle}(k) = k + 2*2^k + 3 + 2 + \frac{3*2^k}{2} + \displaystyle\sum_{i=1}^{2^k} (i) + T_{bucle}(k-1)$$

  que, eliminando la recursividad, es:

  $$ T(k) =  k + 2*2^k + 3 + \displaystyle\sum_{i=1}^k (2 + \frac{3*2^i}{2} + \displaystyle\sum_{j=1}^{2^i} j)$$

  \textit{Nota: esta función está definida en función de $k$ dado que es la que marca la complejidad del problema. La $n$ es consecuente del tamaño de $k$ según la siguiente formula $ n = k + 1 + 2^k $. }

  \vspace{2mm}

  Estos resultados coinciden con los resultados obtenidos empíricamente, que se pueden ver con mayor detalle en la tabla \ref{tab:detalle}.

  \begin{table}[!h]
    \centering
  \begin{tabular}{|c|c|c|c|c|c|c|}
  \hline
  \textbf{Entrada}       & \textbf{Tamaño} & \textbf{k} & \textbf{x} & \textbf{prep(k)} & \textbf{bucle(k)} & \textbf{Total} \\ \hline
  1\$aa                  & 4               &   1        & 2           & 8                & 8                 & 16             \\ \hline
  11\$aaaa               & 7               &   2        & 4           & 13               & 18                & 39             \\ \hline
  111\$aaaaaaaa          & 12              &   3        & 8           & 22               & 50                & 98             \\ \hline
  1111\$aaaaaaaaaaaaaaaa & 21              &   4        & 16          & 39               & 16                & 277            \\ \hline
  \end{tabular}
  \caption{Ampliación con más información sobre cada ejemplo}
  \label{tab:detalle}
\end{table}

Dado que la complejidad está definida sobre $k$, se calcula la cota superior asintótica $g(k)$ usando $k_0 = 10$.

$$ g(k) = c*2^k $$

$$ g(k) > T(k) $$


$$c*2^k >  k + 2*2^k + 3 + \displaystyle\sum_{i=1}^k (2 + \frac{3*2^i}{2} + \displaystyle\sum_{j=1}^{2^i} j) $$

$$c*2^{10} > 10 + 2*2^{10} + 3 + \displaystyle\sum_{i=1}^{10} (2 + \frac{3*2^i}{2} + \displaystyle\sum_{j=1}^{2^i} j) $$

$$ c*2^{10} > 15390 $$

$$ c > \frac{15390}{2^{10}} $$
$$ c >  \frac{7695}{512}$$
$$ c > 15.029 $$

$$ c = 15.03 $$

$$ g(k) = 15.03*2^k $$



\subsection{MT No Determinista de 4 cintas}

\subsubsection{Implementación Propuesta}

El funcionamiento de esta máquina toma como referencia el funcionamiento de la máquina no determinista propuesta en el ejemplo de la práctica, que comprueba si una palabra es un palíndromo.

\vspace{2mm}

Para esta máquina se propone el uso de cuatro cintas, una para llevar a cabo la resta en base 1 del orden $k$ dado, dos cintas más para llevar a cabo la comprobación de la palindromía de la palabra dada, y una última cinta para guardar la mitad de la palabra dada.

\vspace{2mm}

La máquina empieza copiando la palabra $x$ en la cinta 2, dejando solamente $k$ en la cinta 1 y las cintas 3 y 4 vacías. A continuación, se lleva a cabo la comprobación de si la palabra dada es palíndromo de forma no determinista. Para ello, se elimina la primera letra de la palabra, que se encuentra en la cinta 2, y se copia en las cintas 3 y 4. De forma sucesiva, se plantean dos posibles caminos para la máquina: comprobar si las palabras de las cintas 2 y 3 son idénticas, o seguir copiando símbolos en las cintas 3 y 4.

\vspace{1mm}

Si la palabra dada es un palíndromo, cuando la mitad de la palabra ya haya sido borrada de la cinta 2 y copiada en las otras cintas, la máquina comenzará a comprobar que las palabras de las cintas 2 y 3 son iguales (esta transición es tomada cada vez que se copia un nuevo símbolo). Cuando esta comprobación termine, la máquina habrá detectado que la palabra dada es un palíndromo.

\vspace{2mm}

En este punto, en la primera cinta hay uno o varios 1’s, las cintas 2 y 3 están vacías y la cinta 4 almacena la mitad de la palabra original. Cuando esto se da, se resta 1 a $k$ en la primera cinta y se copia la palabra de la cinta 4 a la cinta 2, quedando la cinta 4 vacía. Mientras la cinta 2 no esté vacía, todo el proceso se repite en un bucle, hasta que la palabra analizada no sea un palíndromo o se haya comprobado que la palabra dada es palíndromo de al menos orden $k$.


\begin{figure}[!h]
  \imgcenter[175]{Images/ej4b.png}
  \caption{MT No Determinista de 4 cintas - Comprobación de polinomios de grado $k$}
\end{figure}


\subsubsection{Determinación del Peor Caso}

  \begin{table}[!h]
    \centering
  \begin{tabular}{|c|c|c|}
  \hline
  \textbf{Entrada} & \textbf{Pasos} & \textbf{Resultado} \\ \hline
  \$aaa   & 10             & no acepta          \\ \hline
  \$aab   & 10             & no acepta          \\ \hline
  1\$aa   & 13             & acepta             \\ \hline
  1\$ab            & 11             & no acepta          \\ \hline
  11\$a            & 9              & no acepta          \\ \hline
  111\$            & 7              & no acepta          \\ \hline
  \end{tabular}
  \caption{Peor caso}
  \end{table}

  El peor caso se da, al igual que en la MT determinista, cuando la palabra $x$ es palíndromo de orden $k$ porque es el caso en el que tiene que recorrer más veces el bucle.

  \vspace{3mm}

  En cambio, esta MT, al ser no determinista, no debe aceptar aquellos casos en los que la entrada no cumpla la condición de dada.

\newpage

\subsubsection{Simulación con Diferentes Tamaños}

  \begin{table}[!h]
    \centering
  \begin{tabular}{|c|c|c|}
  \hline
  \textbf{Entrada}       & \textbf{Tamaño} & \textbf{Pasos} \\ \hline
  \$            & 1               & 4              \\ \hline
  1\$aa         & 4               & 13             \\ \hline
  11\$aaaa      & 7               & 29             \\ \hline
  111\$aaaaaaaa          & 12              & 57             \\ \hline
  1111\$aaaaaaaaaaaaaaaa & 21              & 109            \\ \hline
  \end{tabular}
  \caption{Diferente tamaños}
  \end{table}


\subsubsection{Cálculo de $T(n)$}

\begin{table}[!h]
  \centering
\begin{tabular}{|c|p{1cm}|p{1cm}|p{1cm}|p{1cm}|p{1cm}|p{1cm}|}
\hline
\textbf{N}     & \multicolumn{1}{c|}{\textbf{1}} & \multicolumn{1}{c|}{\textbf{4}} & \multicolumn{1}{c|}{\textbf{7}} & \multicolumn{1}{c|}{\textbf{12}} & \multicolumn{1}{c|}{\textbf{21}} \\ \hline
\textbf{Pasos} & \multicolumn{1}{c|}{4}          & \multicolumn{1}{c|}{13}          & \multicolumn{1}{c|}{29}         & \multicolumn{1}{c|}{57}         & \multicolumn{1}{c|}{109} \\ \hline
\textbf{Diferencia 1}                  &                                                         & 9                                                       & 16                               & 28                              &   52                           \\ \hline
\textbf{Diferencia 2}                  &                                                         & \multicolumn{1}{r|}{7}                                  & \multicolumn{1}{r|}{12}          & \multicolumn{1}{r|}{24}       &                         \\ \hline
\end{tabular}
\caption{Diferencias finitas}
\label{tab:difUlt}
\end{table}


Se puede ver en la tabla \ref{tab:difUlt} que por el método de diferencias finitas no se obtienen valores constantes, pero se puede ver que en la diferencia 2 es una exponencial perfecta porque multiplica por 2 cada vez (ignorando el primer caso porque es diferente al resto).

\vspace{2mm}

Se aprecia que $12$ es $ 2^2 *3$ y que $24$ es $2^3 * 3$, por lo que se intenta aproximar con diferentes valores de $k$ y se encuentra que se puede generalizar a $2^k * 3$. Pero como esto no termina de cuadrar con los valores empíricos, se utiliza $2^{k+1} * 3$, que se aproxima más a los valores.


\begin{table}[!h]
  \centering
\begin{tabular}{|c|p{1cm}|p{1cm}|p{1cm}|p{1cm}|p{1cm}|p{1cm}|}
\hline
\textbf{N}     & \multicolumn{1}{c|}{\textbf{1}} & \multicolumn{1}{c|}{\textbf{4}} & \multicolumn{1}{c|}{\textbf{7}} & \multicolumn{1}{c|}{\textbf{12}} & \multicolumn{1}{c|}{\textbf{21}} \\ \hline
\textbf{k}     & \multicolumn{1}{c|}{\textbf{0}} & \multicolumn{1}{c|}{\textbf{1}} & \multicolumn{1}{c|}{\textbf{2}} & \multicolumn{1}{c|}{\textbf{3}} & \multicolumn{1}{c|}{\textbf{4}} \\ \hline
\textbf{Pasos} & \multicolumn{1}{c|}{4}          & \multicolumn{1}{c|}{13}          & \multicolumn{1}{c|}{29}         & \multicolumn{1}{c|}{57}         & \multicolumn{1}{c|}{109} \\ \hline
\textbf{$2^{k+1} * 3$} & \multicolumn{1}{c|}{6}          & \multicolumn{1}{c|}{12}          & \multicolumn{1}{c|}{24}         & \multicolumn{1}{c|}{48}         & \multicolumn{1}{c|}{96} \\ \hline
\textbf{Pasos - $2^{k+1} * 3$} & \multicolumn{1}{c|}{-2}          & \multicolumn{1}{c|}{1}          & \multicolumn{1}{c|}{5}         & \multicolumn{1}{c|}{9}         & \multicolumn{1}{c|}{13} \\ \hline

\textbf{Diferencia 1}                  &                                                         & 3                                                       & 4                               & 4                              &   4                           \\ \hline
\textbf{Diferencia 2}                  &                                                         & \multicolumn{1}{r|}{1}                                  & \multicolumn{1}{r|}{0}          & \multicolumn{1}{r|}{0}       &                         \\ \hline
\end{tabular}
\caption{Ampliación y cálculo de diferencias finitas nuevas}
\label{tab:difUlt2}
\end{table}


En la tabla \ref{tab:difUlt2} se puede ver que haciendo las diferencias finitas a la resta se obtienen finalmente valores constantes en la diferencia 1.



$$ T'(k) = ak + b $$

$$ T'(1) = a + b = 1$$
$$ T'(2) = 2a + b = 5$$

$$ a = 4 , \quad b = -3$$

$$ T'(k) = 4k - 3 $$

Por tanto:

$$ T(k) = 2^{k+1} * 3 + 4k - 3 $$

y coincide con los valores obtenidos empíricamente.

\vspace{2mm}

Dado que la complejidad está definida sobre $k$, se calcula la cota superior asintótica $g(k)$ usando $k_0 = 10$.

$$ g(k) = c*2^k $$

$$ g(k) > T(k) $$

$$ c*2^k > 2^{k+1} * 3 + 4k - 3 $$

$$ c*2^{10} > 2^{11} * 3 + 4*10 - 3 $$

$$ c*2^{10} > 6181 $$
$$ c > \frac{6181}{2^{10}} $$
$$ c > \frac{6181}{1024} $$
$$ c > 6.036 $$

$$ c = 6.037 $$

$$ g(k) = 6.037*2^k $$


\subsection{Evaluación de la mejora obtenida con la MT No Determinista}

\begin{figure}[!h]
  \imgcenter[130]{Images/comparativaMTnoDet.png}
  \caption{Comparativa MT Determinista vs. MT No Determinista}
  \label{fig:comparativaMTNoDet}
\end{figure}

En la figura \ref{fig:comparativaMTNoDet} se observa que el número de pasos que requiere la MT determinista es mucho mayor que la MT no determinista según se aumenta el tamaño de $n$, por lo tanto se puede afirmar que se ha encontrado una mejora con esta última máquina. A pesar de que ambas tengan complejidad exponencial, la máquina determinista tiene un crecimiento mucho más rápido.

\vspace{2mm}

Aún así, el hecho de ser una máquina no determinista tiene una complejidad intrínseca añadida, ya que para encontrar el resultado tiene que realizar una búsqueda entre las posibles acciones que puede realizar la máquina, haciendo que requieran un menor número de pasos para encontrar la solución, pero que los recursos computacionales utilizados para encontrarla sean mayores.

\newpage

\section{Conclusión}

El hecho de utilizar dos cintas en vez de una, consigue reducir la complejidad de las máquinas que poseen una complejidad polinómica de grado 2 a grado 1. Sin embargo, en el caso de aquellas máquinas que presentan una complejidad exponencial, la mejora es menos notable.

\vspace{2mm}

En esta práctica se han intentado diseñar máquinas de Turing de la manera más eficientemente posible, aunque nuestro énfasis ha estado principalmente en encontrar con la mayor precisión posible la complejidad de cada una de ellas. Por eso en algunos casos se ha decidido eliminar o no incorporar distintas mejoras que dificultarían en gran medida el estudio empírico y/o analítico de las máquinas.



\end{document}
